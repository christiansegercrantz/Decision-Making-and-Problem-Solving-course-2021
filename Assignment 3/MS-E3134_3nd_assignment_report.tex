\documentclass{article}
\usepackage[margin=3cm]{geometry}
\usepackage[utf8]{inputenc}
\usepackage{amsmath}
\usepackage{amssymb}
\usepackage{float}
\usepackage{enumitem}
\usepackage{graphicx}
\usepackage{caption}
\usepackage{subcaption}
\usepackage{eurosym}

\usepackage{comment}

\graphicspath{ {Plots/} }


\begin{document}
	\textit{MS-E2134 - Decision making and problem solving}
	\vfill
	{\centering \Huge Assignment 3 \par}
	\vfill
	Christian Segercrantz - 481056 \\
	\par \today
	\pagebreak
	\tableofcontents
	\pagebreak
\section{Attribute-specific value functions - value functions}
The value functions are as follows:
\begin{align}
	v_1(x_1) =& \frac{1}{40}x_1 \\
	v_2(x_2) =& 
		\begin{cases}
		 0,& 0 \leq x_2 \leq 2 \\
		 \frac{1}{12}x_2 -\frac{1}{6},& 2 \leq x_2 < 6 \\
		 \frac{1}{27}x_2 +\frac{1}{9},& 6 \leq x_2 < 15 \\
		 \frac{1}{45}x_2 +\frac{1}{3},& 15 \leq x_2 \leq 30
		\end{cases}\\
	v_3(x_3) =&
		\begin{cases}
			\frac{1}{14}x_3 - \frac{1}{7} & 2 \leq x <9\\
			\frac{1}{42}x_3 + \frac{2}{7} & 9 \leq x \leq 30
		\end{cases} \\
	v_4(x_4) =& \frac{1}{18}x_4 - \frac{1}{9}\\ 
	v_5(x_5) =& \frac{1}{20}x_5\\
	v_6(x_6) =& \frac{1}{20}x_6\\
	v_7(x_7) =& \frac{1}{100}x_7\\
	v_8(x_8) =& 
		\begin{cases}
			\frac{1024}{1023}\left(1-2^{-x_8}\right) & 0 \leq x_8 < 10 \\
			1 &  10 \leq x_8
		\end{cases} \\
	v_9(x_9) =&
		\begin{cases}
			 \frac{1}{4}(1-\sqrt{5})\left(1-\left(\frac{1}{2}\left(1+\sqrt{5}\right)\right)^{3-x_9}\right) & 0 \leq x_9 < 3 \\
			 0 & 3 \leq x_9
		\end{cases}
\end{align}
\begin{comment} v_9 derivation
\begin{equation}
B+A(1-e^{-(3-x_9)/r}) \text{ and } v_9(0)-v_9(1) = v_9(1) - v_9(3)
\end{equation}

\begin{align}
&B+A(1-e^{-(3-x_9)/r})\\ 
v_9(3) = 0 =& B+A(1-e^{-(3-3)/r}) \iff B= 0 \\
v_9(0) = 1 =& A(1-e^{-(3-0)/r}) \\
& \frac{1}{A}= 1 - e^{-3/r} \\
& A= \frac{1}{1-u^3} \\
\end{align}
\begin{align}
v_9(100)= 0 =& A(1-e^{-(3-100)/r})\\
\end{align}
\begin{align}
v_9(0)-v_9(1) &= v_9(1) - v_9(3)\\
2v_9(1) &= v_9(0)\\
v_9(1) &= \frac{1}{2}\\
\frac{1}{2} &= \frac{1}{1-u^3}(1-u^{2})\\
1-u^3 &= 2-2u^2\\
-u^3 + 2u^2 - 1 &= 0 \implies  u=\frac{1}{2}+\frac{\sqrt{5}}{2} \left(\lor u=1 \lor u=\frac{1}{2}-\frac{\sqrt{5}}{2}\right) \\
e^{-1/r} &= 	\frac{1}{2}+\frac{\sqrt{5}}{2} \iff r = \frac{1}{ln(2) - ln(1 + sqrt(5))}	\\
\implies A&= \frac{1}{1-( 1/2 + sqrt(5)/2)} \iff A= \frac{1}{4}(1-\sqrt{5}) \\
\implies v_9&= \frac{1}{4}(1-\sqrt{5})\left(1-\left(\frac{1}{2}\left(1+\sqrt{5}\right)^{-(3-x_9)}\right)\right)
\end{align}
\end{comment}
\section{Attribute-specific value functions - values}
		The attribute values for all alternatives can be found from the attached excel.
\section{Attribute-specific value functions - value plots}
	\begin{figure}[H]
		\includegraphics[width=0.8\textwidth]{3_v2.png}
		\caption{The plot for $v_2$.}
		\label{fig:3_v2}
	\end{figure}
	\begin{figure}[H]
		\includegraphics[width=0.8\textwidth]{3_v9.png}
		\caption{The plot for $v_9$.}
		\label{fig:3_v9}
	\end{figure}
\section{Attribute weights}
	From 1. , and knowing that 20 is the max and 0 the min for both value functions, we get the equation
	\begin{align}
		\frac{w_5 v_5(20)-w_5 v_5(0)}{w_6v_6(20)-w_6v_6(0)} &= \frac{40}{45}\\
		w_5 = \frac{8}{9}w_6.
	\end{align}
	From 2., and knowing that the max value is 20 and min value is 2, we get
	\begin{align}
		\frac{w_6 v_6(20)-w_6 v_6(0)}{w_4v_4(20)-w_4v_4(2)} &= \frac{45}{30}.\\
		w_4 = \frac{2}{3}w_6.
	\end{align}
	From 3. we get information about two equally preferred preferences, and again remembering the min value of the value functions,
	\begin{align}
		w_8v_8(1) - w_8v_8(0) &= w_4v_4(10) - w_4v_4(2)\\
		w_8 &=  \frac{v_4(10)}{v_8(1)}w_4.
	\end{align}
	From 4. we get a similar equation as above
	\begin{align}
		w_7v_7(1) - w_7v_7(0) &= w_4v_4(3) - w_4v_4(2)\\
		w_7 &=  \frac{v_4(3)}{v_7(1)}w_4.
	\end{align}
	From 5.	get multiple the following equality
	\begin{align}
		w_1v_1(40) - w_1v_1(0) &= w_2v_2(30) - w_2v_2(0) + w_3v_3(20) - w_3v_3(0)\\
		w_1 &= w_2 + w_3v_3(20).
	\end{align}
	In 6., we know that all but two values do not change, hence we can omit them from the start
	\begin{align}
		w_1v_1(40) + w_9v_9(1.2) &= w_1v_1(10) + w_9v_9(0)\\
		w_1 + w_9v_9(1.2) &= w_1v_1(10) + w_9\\
		w_1(1-v_1(10))&=w_9(1-v_9(1.2))\\
		w_9&=\frac{1-v_1(10)}{1-v_9(1.2)}w_1.
	\end{align}
	In 7., we know that the two changes from the minimum $x^0$ are equally preferred, i.e
	\begin{align}
		w_4v_4(10) + w_9v_9(1.2) - w_4v_4(2) - w_9v_9(100) &= w_4v_4(18) + w_9v_9(3) - w_4v_4(2) - w_9v_9(100)\\
		w_4v_4(10) + w_9v_9(1.2) &= w_4v_4(18) + w_9v_9(3)\\
		w_4v_4(10) + w_9v_9(1.2) &= w_4v_4(18)\\
		w_9 = \frac{v_4(18)-v_4(10)}{v_9(1.2)}w_4.
	\end{align}
	From 8. we get
	\begin{align}
		w_2v_2(15)-w_2v_2(0) &= w_3v_3(30)-w_3v_3(2)\\
		w_2v_2(15) &= w_3.
	\end{align}
	Additionally we know that the sum of weights equals to one 
	\begin{equation}
		\sum_{i=1}^{9}w_i=1.
	\end{equation}
	By solving these equations, we get the following values
	\begin{alignat}{3}
		w_1&= 0.062131...& \approx 0.06,\\
		w_2&= 0.0412027...&\approx0.04,\\
		w_3&= 0.0274685...&\approx0.03,\\
		w_4&= 0.076779...&\approx0.08,\\
		w_5&= 0.102372...&\approx0.10,\\
		w_6&= 0.115168..&\approx0.12,\\
		w_7&= 0.42655...&\approx0.43,\\
		w_8&= 0.0681814...&\approx0.07,\\
		w_9&= 0.080147...&\approx0.08.
	\end{alignat}
\section{Overall values}
	Table \ref{tab:ex5} displays the normalized values of the sites times the area.
	\begin{table}[H]
		\centering
		\caption{Normalized vlaue functions of the sites times the area.}
		\label{tab:ex5}
		\begin{tabular}{lll}
			\textbf{Site} & \textbf{Area} & \textbf{Normalizied value} \\ \hline
			1             & 1,2           & 0,26                       \\
			2             & 3             & 0,88                       \\
			3             & 2,1           & 0,71                       \\
			4             & 3             & 0,64                       \\
			5             & 0,8           & 0,25                       \\
			6             & 2             & 0,70                       \\
			7             & 3             & 0,70                       \\
			8             & 0,9           & 0,23                       \\
			9             & 1,1           & 0,37                       \\
			10            & 2,4           & 0,48                      
		\end{tabular}
	\end{table}
\section{Recalculated overall values}
	Since $a_7$ most preferred level changes from 100 to 40, the value function  becomes $v_7=\frac{1}{40}x_7$. Thus, the only value in our system of equations for solving the weights that change is $v_7(1)$. The recalculated weights are
	\begin{alignat}{3}
		w_1&= 0.0835016...& \approx 0.08,\\
		w_2&= 0.0553747...&\approx0.06,\\
		w_3&= 0.0369165...&\approx0.04,\\
		w_4&= 0.103188...&\approx0.10,\\
		w_5&= 0.137584...&\approx0.14,\\
		w_6&= 0.154782..&\approx0.15,\\
		w_7&= 0.229306...&\approx0.23,\\
		w_8&= 0.091633...&\approx0.09,\\
		w_9&= 0.107714...&\approx0.11.
	\end{alignat}
	and the scores can be sen in Table \ref{tab:ex6}.
	\begin{table}[]
		\centering
		\caption{Normalized value functions of the sites times the area.}
		\label{tab:ex6}
		\begin{tabular}{lll}
			\textbf{Site} & \textbf{Area} & \textbf{Normalizied value} \\ \hline
			1             & 1,2           & 0,34                       \\
			2             & 3             & 1,18                       \\
			3             & 2,1           & 0,95                       \\
			4             & 3             & 0,86                       \\
			5             & 0,8           & 0,34                       \\
			6             & 2             & 0,94                       \\
			7             & 3             & 0,94                       \\
			8             & 0,9           & 0,31                       \\
			9             & 1,1           & 0,50                       \\
			10            & 2,4           & 0,65                      
		\end{tabular}
	\end{table}

	If we calculate $\frac{V_i'(x_i)}{V_i(x_i)}$ we get a constant $\alpha\approx1.34$ for all $i$. This suggest that $V_i'(x_i)=\alpha V_i(x_i)$ for all $i$, i.e. the $V'$ is a affine transformation of $V$. 
	
\section{Site combination selection}
	The binary linear optimization problem we need to solve is
	\begin{alignat}{3}
		\max_y& \sum_{j=1}^{10} V(x^j)y_j\\
		\text{subject to} & \sum_{j=1}^{10} c_jy_j &\leq 25000\\
		&y_j \in \mathbb{B},
	\end{alignat}
	where b is a binary variable indicating if a site is to be acquired. 
	
	The calculations for this part is done in the attached excel. The optimal solution are the following sites: 2, 3, 5, 6, 7, 9. This gives a optimal value of approximately $4,85$ at a cost of 24494\euro.

\section{Multi-objective optimization applied to site combination selection - optimization}

	
\end{document}
\documentclass{article}
\usepackage[margin=3cm]{geometry}
\usepackage[utf8]{inputenc}
\usepackage{amsmath}
\usepackage{amssymb}
\usepackage{float}
\usepackage{enumitem}
\usepackage{graphicx}
\usepackage{caption}
\usepackage{subcaption}
\usepackage{eurosym}

\usepackage{comment}

\graphicspath{ {Plots/} }


\begin{document}
	\textit{MS-E2134 - Decision making and problem solving}
	\vfill
	{\centering \Huge Assignment 3 \par}
	\vfill
	Christian Segercrantz - 481056 \\
	\par \today
	\pagebreak
	\tableofcontents
	\pagebreak
\section{Attribute-specific value functions - value functions}
The value functions are as follows:
\begin{align}
	v_1(x_1) =& \frac{1}{40}x_1 \\
	v_2(x_2) =& 
		\begin{cases}
		 0,& 0 \leq x_2 \leq 2 \\
		 \frac{1}{12}x_2 -\frac{1}{6},& 2 \leq x_2 < 6 \\
		 \frac{1}{27}x_2 +\frac{1}{9},& 6 \leq x_2 < 15 \\
		 \frac{1}{45}x_2 +\frac{1}{3},& 15 \leq x_2 \leq 30
		\end{cases}\\
	v_3(x_3) =&
		\begin{cases}
			\frac{1}{14}x_3 - \frac{1}{7} & 2 \leq x <9\\
			\frac{1}{42}x_3 + \frac{2}{7} & 9 \leq x \leq 30
		\end{cases} \\
	v_4(x_4) =& \frac{1}{18}x_4 - \frac{1}{9}\\ 
	v_5(x_5) =& \frac{1}{20}x_5\\
	v_6(x_6) =& \frac{1}{20}x_6\\
	v_7(x_7) =& \frac{1}{100}x_7\\
	v_8(x_8) =& 
		\begin{cases}
			\frac{1024}{1023}\left(1-2^{-x_8}\right) & 0 \leq x_8 < 10 \\
			1 &  10 \leq x_8
		\end{cases} \\
	v_9(x_9) =&
		\begin{cases}
			 \frac{1}{4}(1-\sqrt{5})\left(1-\left(\frac{1}{2}\left(1+\sqrt{5}\right)\right)^{3-x_9}\right) & 0 \leq x_9 < 3 \\
			 0 & 3 \leq x_9
		\end{cases}
\end{align}
\section{Attribute-specific value functions - values}
\section{Attribute-specific value functions - value plots}
\section{Attribute weights}
\begin{comment} v_9 derivation
\begin{equation}
	B+A(1-e^{-(3-x_9)/r}) \text{ and } v_9(0)-v_9(1) = v_9(1) - v_9(3)
\end{equation}

\begin{align}
	&B+A(1-e^{-(3-x_9)/r})\\ 
	v_9(3) = 0 =& B+A(1-e^{-(3-3)/r}) \iff B= 0 \\
	v_9(0) = 1 =& A(1-e^{-(3-0)/r}) \\
	& \frac{1}{A}= 1 - e^{-3/r} \\
	& A= \frac{1}{1-u^3} \\
\end{align}
\begin{align}
	v_9(100)= 0 =& A(1-e^{-(3-100)/r})\\
\end{align}
\begin{align}
	v_9(0)-v_9(1) &= v_9(1) - v_9(3)\\
	2v_9(1) &= v_9(0)\\
	v_9(1) &= \frac{1}{2}\\
	\frac{1}{2} &= \frac{1}{1-u^3}(1-u^{2})\\
	1-u^3 &= 2-2u^2\\
	-u^3 + 2u^2 - 1 &= 0 \implies  u=\frac{1}{2}+\frac{\sqrt{5}}{2} \left(\lor u=1 \lor u=\frac{1}{2}-\frac{\sqrt{5}}{2}\right) \\
	e^{-1/r} &= 	\frac{1}{2}+\frac{\sqrt{5}}{2} \iff r = \frac{1}{ln(2) - ln(1 + sqrt(5))}	\\
	\implies A&= \frac{1}{1-( 1/2 + sqrt(5)/2)} \iff A= \frac{1}{4}(1-\sqrt{5}) \\
	\implies v_9&= \frac{1}{4}(1-\sqrt{5})\left(1-\left(\frac{1}{2}\left(1+\sqrt{5}\right)^{-(3-x_9)}\right)\right)
\end{align}
\end{comment}
\end{document}